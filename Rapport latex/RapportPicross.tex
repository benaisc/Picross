\documentclass{article}
\usepackage{hyperref}
\usepackage{xcolor}
\begin{document}

\title{Rapport de projet Picross}
\author{CONNES Victor, saississez vos noms}
\maketitle
\tableofcontents 
\newpage
\section{Description de l'impl\'ementation}
\subsection{ Les besoins du programme :}

\subsubsection{Pour les indices de chaque ligne/colonne:}
\begin{itemize}
\item Imperatif \`a satisfaire:
\begin{itemize}

\item structure de taille dynamique (nombre variable d’indice par ligne)
\item parcours en $\theta$(n) (plusieurs m\'ethodes recourant \`a  un parcours total ou partiel de la structure)
\item relation d’ordre (le parcours ne sont r\'ealiser que dans l’ordre de l’indice le plus en haut vers celui le plus en bas respectivement gauche-droite pour les
lignes)
\end{itemize}
\item Choix : la liste simplement chain\'es, car elle correspond parfaitement au sp\'ecification et parait finalement tr\`es proche de la r\'ealit\'e
\end{itemize}
\subsubsection{Pour repr\'esenter la grille du picross :}
\begin{itemize}
\item Imperatif \`a satisfaire:
\begin{itemize}
\item structure de taille fixe
\item acc\`es en temps constant a chaque case
\item faciliter de copie de la structure ($\theta$(n))
\end{itemize}
\item Choix: La matrice car tr\`es proche de la r\'ealit\'e
\end{itemize}
\subsubsection{Pour l’ensemble des indices de lignes/colonnes :}
\begin{itemize}
\item Imperatif \`a satisfaire:
\begin{itemize}
\item acc\'eder en temps constant a chaque liste
\item garder un indicage coh\'erent avec la matrice
\end{itemize}
\item Choix :Un tableau
\end{itemize}
\subsubsection{Pour ligmodif /colmodif}
\begin{itemize}
\item Imperatif \`a satisfaire:
\begin{itemize}
\item ajout d’un \'el\'ement en temps constant
\item retrait d’un \'el\'ement en temps constant
\item taille dynamique
\end{itemize}
\item Choix : 
Le choix naturel serait l’ensemble mais on choisit la liste simplement chain\'es car d\'ej\`a  impl\'ement\'es pour les indices de chaque ligne/colonne. Son
comportement est similaire dans le cas d'un ajout et d'un retrait en d\'ebut de liste .
\end{itemize}
\subsection{ Les classes}
L'ensemble de nos classes disposent d'une surcharge de l'operateur<< afin de faciliter l'affichage dans le terminal. De plus, il existe d'autres classes dans notre
programme qui permettent d'am\'eliorer l'affichage en utilisant une interface graphique. Nous ne distinguerons ici seulement les classes permettant la
resolution du picross.

\subsubsection{La classe Cell :}
Elle repr\'esente un indice logique
\begin{itemize}
\item Attributs :
\begin{itemize}
\item Val : qui repr\'esente la valeur de cette indice logique (cette valeur \'etant strictement enti\`ere positive et a priori non born\'e nous avons choisit de la repr\'esenter par un size\_t)
\item Suiv : qui est pointeur sur la cellule suivante de la liste
\end{itemize}
\end{itemize}
\subsubsection{La classe Liste :}
\begin{itemize}
\item Attributs :
\begin{itemize}
\item Longueur : qui repr\'esente la longueur de la liste. Cette attribut et incr\'ementer ou d\'ecr\'ementer automatiquement d\`es qu'il y a variations de la taille de la
liste.
\item Fini : Nous indique si la ligne/colonne a laquelle est rattach\'e la liste est enti\`erement rempli (bool\'een \`a  true) ou non (bool\'een \`a  false), cette attribut et
notamment tr\`es utile pour v\'erifier la condition d'arrêt de notre r\'esolution (cf. main)
\item Tête : Pointeur vers le premier \'el\'ement de la liste.
\end{itemize}
\item M\'ethode remarquable :
\begin{itemize}
\item Surcharge de l'op\'erateur() : Nous avons utiliser l'operateur() comme un operateur d'indexage d'une liste cela nous permet notamment de faciliter le
parcours d'une liste.
\end{itemize}
\end{itemize}
\subsubsection{La classe Tabliste :}
\begin{itemize}
\item Attributs :
\begin{itemize}
\item Tab : le tableau de liste
\item Taille : la taille du tableau
\end{itemize}
\item M\'ethode remarquable :
\begin{itemize}
\item Surcharge de l'op\'erateur[]: Nous permet d'utiliser comme si elle \'etait simplement un tableau.
\end{itemize}
\end{itemize}
\subsubsection{La classe Matrice :}
\begin{itemize}
\item Attributs:
\begin{itemize}
\item Tab : la matrice d'entier (-1 : blanc 0 : ind\'etermin\'e 1 : noir)
\item Nbl : Nombre de ligne
\item Nbc : Nombre de colonnes
\end{itemize}
\end{itemize}
\subsection{ La classe Picross :}
La classe picross est la classe dont les instances repr\'esente les picross c'est dans cette classe que sont impl\'ement\'e nos m\'ethodes de r\'esolution.
\subsection{Les algorithmes remarquables :}
\subsubsection{SLG :}
\paragraph{Sp\'ecification :}
\paragraph{Complexit\'e pire des cas : $\theta$($n^2$)}
\paragraph{Complexit\'e meilleur des cas : $\theta$(n)}
\end{document}